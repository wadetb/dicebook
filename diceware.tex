\chapter*{Secure passphrases with Diceware}
\label{ch:diceware}

\vspace{-.2in}
\includegraphics[width=\textwidth]{password_strength.png}

\vspace{-.125in}
\begin{tiny}
	\begin{center}
		Comic by \url{https://xkcd.com/} \ccbync 
	\end{center}
\end{tiny}

\newpage
\setlength{\parindent}{0em}
\setlength{\parskip}{0.25em}

\subsubsection*{What is a Diceware passphrase?}

A passphrase is a bunch of words and characters that you type into your computer to let it know for sure that \textit{the person typing is you}. Passphrases differ from passwords only in length.

Diceware is a method for picking passphrases which uses dice to select words at random from a special list. Each word in the list is preceded by a four digit number. All the digits are between one and six, allowing you to use the outcomes of four dice rolls to select a word from the list.

\subsubsection*{Directions}

\begin{itemize}[leftmargin=*]

\item[1] Decide how many words you want in your passphrase. A five-word passphrase provides a level of security \textit{much} higher than the passwords most people use. For NSA-proof security, use seven words.

\item[2] Roll all four dice at once and write the results down, reading from left to right, on a scrap of paper. Make as many of these four-digit groups as you want words in your passphrase.

\item[3] Flip to the word list and look up the corresponding word next to each four-digit number.

\item[4] The words you have found are your new passphrase!

\item[5] Come up with a way to remember your phrase. It might be a story, mental picture, or sentence that will remind you of the phrase you generated.

\end{itemize}

\newpage

\setlength{\parskip}{0.5em}

\subsubsection*{Example}

Suppose you want a five-word passphrase. You will need $5 \times 4$ or 20 dice rolls. Let's say they come out as:

\begin{adjustwidth}{2em}{2em}
\makebox[\linewidth][s]{\textit{1 6 6 6 5 1 5 6 5 3 5 6 3 2 2 3 5 6 2 6}}
\end{adjustwidth}

Write the results on a scrap of paper in groups of four:

\begin{adjustwidth}{2em}{2em}
\textit{1 6 6 6 \\
5 1 5 6 \\
5 3 5 6 \\
3 2 2 3 \\
5 6 2 6}
\end{adjustwidth}

You then look up each group of five rolls in the Diceware word list by finding the number in the list and writing down the word next to the number:

\begin{adjustwidth}{2em}{2em}
\textit{1666 cognitive \\
5156 poetic \\
5356 ribbon \\
3223 garlic \\
5626 smelliness}
\end{adjustwidth}

Your passphrase would then be:

\begin{adjustwidth}{2em}{2em}
\textit{cognitive poetic ribbon garlic smelliness}
\end{adjustwidth}

This passphrase is one of the 3,656,158,440,062,976 ($6^{20}$) alternatives that could have been chosen by this method.

%(\approx$2^{51}$) 

\setlength{\parskip}{0.25em}
